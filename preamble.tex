\usepackage[a4paper,top=2cm, bottom=2cm, left=3cm, right=1.5cm]{geometry} %Настройка полей документа
\linespread{1.5}
\usepackage{xecyr}
\usepackage{mathtext}
\usepackage{amsmath, amsfonts, amssymb}%Математические вкусности (нумерованные формулы)
\usepackage{mathtools}%Прямое указание типа дробей и прочее
\usepackage{etoolbox}
\usepackage[utf8]{inputenc}%включаем свою кодировку: koi8-r или utf8 в UNIX, cp1251 в Windows
\usepackage[english,russian]{babel}	
\usepackage[final,expansion=true]{microtype}
\setmainfont{Times New Roman}%Основной шрифт
\setsansfont{CMU Sans Serif}%Шрифт без засечек (для выделения)
\setmonofont{CMU Typewriter Text}%Моноширинный шрифт (Для кода)
\DeclareSymbolFont{letters}{\encodingdefault}{\rmdefault}{m}{m}%Магия прямого шрифта в индексах и кирилиццы в них же. Для замены прямого на курсив, поменять вторую m на it
\usepackage{graphicx}%Вставка картинок правильная
\graphicspath{ {./images/} }%путь к папки с картинками
\usepackage{wrapfig}%Обтекание фигур (таблиц, картинок и прочего)
\usepackage[export]{adjustbox}%Обрезка, подгонка картинок
\usepackage{longtable}%Многостраничные таблицы
\RequirePackage{caption}
\DeclareCaptionLabelSeparator{defffis}{ --- } %Разделитель в подписях к рисункам
\captionsetup{justification=centering,labelsep=defffis}
\usepackage{subfig} % Подкартинки, с собственными подписями и ссылками
\usepackage{tikz}%Встроенная рисовалка
\usetikzlibrary{calc,shapes,arrows,chains,fit}%Библиотеки для рисовалки. DSP требует установки отдельно http://www.texample.net/tikz/examples/fir-filter/
\usepackage{pgfplots}%Встроенная чертилка. Работает внутри рисовалки.
\tikzset{every picture/.style={line width=1pt}}%Смена толщины линий на более толстые
\tikzset{
	ultra thin/.style= {line width=0.2pt},
	very thin/.style= {line width=0.4pt},
	thin/.style= {line width=1pt},% 0.4 thin is the default
	semithick/.style= {line width=1.3pt},
	thick/.style= {line width=1.6pt},
	very thick/.style= {line width=2pt},
	ultra thick/.style={line width=2.5pt}
}
\usepackage{pgfplotstable}%Черчение данных из файлов
\usepgfplotslibrary{groupplots}%Множественные графики. График над графиком.
\usetikzlibrary{external} %Ускоряет сборку документов с картинками (?)
%\pagenumbering{gobble}%Когда включено, отключает нумерацию страниц
\usepackage{lscape}%Вставка страниц в ландшафтной ориентации
\usepackage{pdflscape}%Корректная работа ландафтных страниц в PDF
\usepackage{hyperref} %Поддержка ссылок в PDF. Делает все ссылки в PDF рабочими
\usepackage[russian]{cleveref} %Умные ссылки -- \cref{fig:123} ссылается на картинку, "рис. 123". Умеет ссылки на список, причем умные
\usepackage{indentfirst} % отделять первую строку раздела абзацным отступом
\setlength\parindent{5ex}
\usepackage{url}

\usepackage[tableposition=top,singlelinecheck=false, justification=centering]{caption}
\usepackage{subcaption}
%  маркированные списки
\renewcommand{\labelitemi}{--}
\renewcommand{\labelitemii}{--}
%  нумерованные списки
\renewcommand{\labelenumi}{\asbuk{enumi})}
\renewcommand{\labelenumii}{\arabic{enumii})}

% номер сноски со скобкой
\renewcommand*{\thefootnote}{\arabic{footnote})}
\renewcommand{\footnoterule}{%
	\kern -3pt
	\hrule width 40mm height .4pt
	\kern 2.6pt
}


\renewcommand{\figurename}{Рисунок} % Рис -> Рисунок
\usepackage[labelfont=normalfont,figurename={Рисунок},figurewithin=none]{caption}
%иллюстрации и таблицы
\DeclareCaptionLabelFormat{gostfigure}{Рисунок #2}
\DeclareCaptionLabelFormat{gosttable}{Таблица #2}
\DeclareCaptionLabelSeparator{gost}{~---~}
\captionsetup{labelsep=gost}
\captionsetup*[figure]{labelformat=gostfigure}
\captionsetup*[table]{labelformat=gosttable}
\renewcommand{\thesubfigure}{\asbuk{subfigure}}

\renewcommand{\theenumi}{\arabic{enumi}}% Меняем везде перечисления на цифра.цифра
\renewcommand{\labelenumi}{\arabic{enumi}}% Меняем везде перечисления на цифра.цифра
\renewcommand{\theenumii}{.\arabic{enumii}}% Меняем везде перечисления на цифра.цифра
\renewcommand{\labelenumii}{\arabic{enumi}.\arabic{enumii}.}% Меняем везде перечисления на цифра.цифра
\renewcommand{\theenumiii}{.\arabic{enumiii}}% Меняем везде перечисления на цифра.цифра
\renewcommand{\labelenumiii}{\arabic{enumi}.\arabic{enumii}.\arabic{enumiii}.}% Меняем везде перечисления на цифра.цифра




\usepackage{listings}
\lstset{
	frame=single,
	breaklines=true
}

\usepackage[normalem]{ulem}